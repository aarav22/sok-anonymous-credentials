Idemix \cite{idemix} is an implementation of an anonymous credential system based on protocols described in \cite{Camenisch2001AnES}. The system offers an easy-to-use access control mechanism that can be integrated with user applications. The library needs to be imported into the application, and the application extends the library with its own access control policies.

For instance, the authors illustrate the system's utility in the context of a New York Times (NYT) news service subscription. In this example, a user first acquires a credential from a bank (ARGENTIX) for \$10, which is then shown to the New York Times news subscription service (KIOSK) to obtain another credential. The final credential is presented to NYT to get the subscription. Throughout these transactions, pseudonyms are used, and the process is unlinkable, thereby ensuring full anonymity for the user.

However, the system still requires an external trusted third party to issue root pseudonyms and credentials based on some verification of the user's real identity. To prevent credential sharing, the authors propose two types of safeguards: \textit{all-or-nothing}, where the user can share all or none of their credentials, and \textit{PKI-assured non-transferability}, which links a master secret to a valuable secret key from outside the system (e.g., he secret key that gives access to the user's bank account). The latter option is rarely available.

Optional anonymity revocation also requires a third party to reveal the pseudonym used for the transaction (local revocation) or reveal the user's real identity (global revocation). Both the user and the organization need to agree on the conditions that would lead to revocation (e.g., illegal transactions).

The system's implementation uses zk-proofs to validate user credentials, resulting in significant computational overhead on both the user and organization sides. The authors' benchmark results show that registering a pseudonym, issuing a credential, and showing a credential can take up to 25 seconds.
