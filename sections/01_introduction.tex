Data minimization as an abstract strategy describes the avoidance of unnecessary or unwanted data disclosures. The most fundamental information that can be disclosed about an individual is who he is (an identifier, or which observable events he is related to) \cite{FISCHERHBNER2017759}. If this information can be kept secret, the individual remains anonymous. Pfitzmann and Hansen, who pioneered the technical privacy research terminology, define anonymity as follows: Anonymity of a subject means that the subject is not identifiable within a set of subjects, the anonymity set \cite{anon_terminology}. 

The subject here is any entity defined by facts (names or identifiers), or causing observable events (by sending messages) while if an adversary cannot narrow down the sender of a specific message to less than two possible senders, the actual sender of the message remains anonymous. The two or more possible senders in question form the anonymity
set.

Data minimization can be achieved through obfuscating of facts and events (messages, emails, and actions). If the attacker cannot detect confidential materials then they cannot link them to the subjects. This is defined as undectability \cite{anon_terminology}. We can achieve data minimization by removing any relation between the subject and the confidential materials. This is called unlinkability \cite{anon_terminology}. We discuss another form of data minimization through the use of psuedonyms. Introduced by Chaum \cite{Chaum1985}, pseudonymity is related to anonymity as both concepts aim at protecting the real identity of a subject. The use of pseudonyms, however, allows to maintain a reference to the subject's real identity.

We structure the SoK by first discussing the history of anonymous credentials, followed by a discussion of the design of anonymous credentials, and finally, we discuss the applications of anonymous credentials.

% In the history section, we discuss the evolution of anonymous credentials from the early days of cryptography to the present day. This includes the ideas of anonymous credetials by Chaum \cite{Chaum1985}. In the design section, we discuss the various design choices that are made when designing anonymous credential systems. In the applications section, we discuss the various applications of anonymous credentials.